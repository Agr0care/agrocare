\section{Marco Teórico}

Se ha comprobado que la tecnología es muy útil para el cuidado y seguimiento de las plantas. Actualmente, existen diversas tecnologías como los sensores de temperatura y humedad que permiten monitorear y controlar el entorno de las plantas.
\vspace{0.5cm}

Estos sensores pueden conectarse a dispositivos de Internet de las Cosas (IoT), lo que facilita la recolección y análisis de datos en tiempo real, permitiendo tomar decisiones y establecer estrategias para el cuidado de las plantas.
\vspace{0.5cm}

Existen aplicaciones móviles y plataformas en línea que permiten monitoreo y control remoto de los huertos, aprovechando la implementación del IoT. La información recolectada puede ser guardada y procesada por una Red Neuronal Recurrente (RNN) para una proyección de datos basada en datos históricos. La implementación de tecnología en los huertos puede mejorar la productividad y calidad de los cultivos, además de tener un impacto positivo en el medio ambiente. Para este propósito, se determinó que IoT y RNN son útiles ya que el primero envía datos en tiempo real y los usuarios pueden interactuar con ellos, mientras que el segundo ajusta los sesgos utilizando data histórica. Además, el uso de RNN asegura la fidelidad de los datos ya que es utilizada para datos secuenciales, como es el caso de Agrocare.
\vspace{0.5cm}

Existe un trabajo de investigación acerca de la robótica e inteligencia artificial para producción "\textit{Robotics for plant proudction}" \citep{robotics_for_plant_production}, sin embargo este se enfoca en producción masiva por lo que no es viable enfocarlo a huertos urbanos, a pesar de esto, la investigación aporta datos que ayudan a entender cómo es que un sistema de este estilo se comportaría en diferentes ambientes dados sus factores ambientales (algunos no controlables, según el estudio), y culturales, dando pie a la pregunta de si es o no viable utilizar estos sistemas en producción.