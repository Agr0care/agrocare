\section{Glosario}
\noindent\textbf{Inteligencia Artificial} (IA): Capacidad de máquinas y sistemas informáticos para realizar tareas mediante programación de algoritmos y técnicas de aprendizaje automático.

\noindent\textbf{Internet de las Cosas} (IoT): Interconexión de dispositivos físicos a través de Internet para interactuar con el mundo real y comunicarse entre sí.

\noindent\textbf{Microcontrolador}: Dispositivo programable que integra procesador, memoria y periféricos de entrada/salida, diseñado para controlar procesos específicos en tiempo real.

\noindent\textbf{Sensor}: Dispositivo que mide una magnitud física o química y convierte la información en una señal eléctrica o digital.

\noindent\textbf{Fotorresistencia} (LDR): Tipo de sensor que varía su resistencia eléctrica según la cantidad de luz que recibe, utilizado en sistemas de iluminación y control de exposición en fotografía.

\noindent\textbf{Capacitivo}: Tecnología de sensores que miden la capacidad de almacenamiento de carga eléctrica en un material dieléctrico, utilizados en pantallas táctiles y botones capacitivos.

\noindent\textbf{Red Neuronal Recurrente}: Modelo de inteligencia artificial de aprendizaje profundo utilizado en el procesamiento de secuencias de datos, como el reconocimiento del habla y el procesamiento del lenguaje natural.

\noindent\textbf{API}: Conjunto de reglas y protocolos que permiten la interacción y comunicación entre diferentes sistemas o aplicaciones de software, utilizadas para la integración de servicios y la automatización de tareas.

\noindent\textbf{\textit{Socket}}: Mecanismo de comunicación entre procesos en una red de computadoras, permitiendo el intercambio de información mediante protocolos de comunicación estándar.

\noindent\textbf{LCD}: Tipo de pantalla de visualización que utiliza cristales líquidos para mostrar imágenes y texto, comúnmente utilizadas en dispositivos electrónicos portátiles.

\noindent\textbf{Déficit Hídrico}: Condición adversa a la que se expone un cultivo cuando no hay agua suficiente.

\noindent\textbf{Estrés Hídrico}: Condición adversa a la que se expone un cultivo cuando el suelo está excesivamente húmedo.