\section{Definición del Problema}

\subsection{Huertos urbanos}

Los huertos urbanos son jardines o cultivos a pequeña escala que buscan producir plantas, frutas y verduras en hogares \citep{gobmx_huertos_urbanos}. Sin embargo, el estilo de vida acelerado y ocupado de las personas suele impedir el cuidado adecuado de las plantas, lo que reduce la producción y puede incluso llevar a la muerte de las plantas. A pesar de esto, los huertos urbanos han ganado popularidad en los últimos años debido a los efectos económicos de la pandemia por Covid-19 y la conciencia ambiental. En Guadalajara, se han creado organizaciones civiles para fomentar la creación y mantenimiento de huertos barriales en la ciudad \citep{huertos_urbanos_gdl}. Sin embargo, esta tendencia no sólo se ha visto en crecimiento en Guadalajara, también en el resto del país, es así que la Ciudad de México se posiciona como la segunda con mayor cantidad de huertos urbanos en latinoamérica \citep{huertos_urbanos_cdmx}.

\subsection{Huertos rurales}

A medida que los huertos rurales crecen en tamaño, surgen desafíos en el cuidado y mantenimiento de los cultivos. Uno de los principales desafíos es la falta de atención constante, lo que es esencial para el cuidado de los cultivos, pero a menudo resulta complicado y costoso. Existen dispositivos tecnológicos para el cuidado y monitoreo de los huertos, pero su alto costo los hace inaccesibles para muchos productores nuevos en la industria. Estos desafíos afectan tanto a los productores como a los consumidores, ya que los precios elevados de los alimentos pueden dificultar el acceso a productos saludables y nutritivos, lo que a su vez afecta la calidad de vida de la población.

Es fundamental encontrar soluciones asequibles y eficaces para los productores urbanos y rurales que permitan mantener la productividad de los huertos sin elevar los precios de los alimentos.