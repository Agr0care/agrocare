\section{Conclusiones}
En conclusión, los huertos urbanos y rurales se han vuelto una tendencia en la producción de alimentos debido a la conciencia ambiental y los efectos económicos de la pandemia por Covid-19. Sin embargo, su cuidado adecuado es un desafío, especialmente en el caso de los huertos rurales debido a la falta de atención constante y el alto costo de la tecnología de monitoreo. Es fundamental encontrar soluciones asequibles y eficaces para los productores y consumidores, lo que mejoraría la productividad y calidad de los cultivos, y reduciría el impacto negativo en el medio ambiente.\\

La tecnología actual, como los sensores de temperatura y humedad conectados a dispositivos de IoT, puede ser utilizada para el monitoreo y control remoto de los huertos. La implementación de tecnología en los huertos puede mejorar la productividad y calidad de los cultivos, además de tener un impacto positivo en el medio ambiente. Además, la recolección y análisis de datos en tiempo real permiten tomar decisiones y establecer estrategias para el cuidado de las plantas. El uso de Redes Neuronales Recurrentes permite ajustar los sesgos utilizando datos históricos y asegura la fidelidad de los datos, lo que resulta útil en proyectos como Agrocare.\\

El objetivo del proyecto es determinar si el cuidado de plantas automatizado con robótica es eficiente, escalable y sostenible a largo plazo. Con la ayuda de Agrocare, se espera medir los valores que afectan el crecimiento de una planta, incluyendo la humedad del suelo, humedad relativa, luz solar y temperatura. Posteriormente, estos datos serán analizados en un servidor remoto apoyándose de un microservicio de Red Neuronal Recurrente y una base de datos. El flujo de datos será automatizado y eficiente, lo que permitirá que el cuidado de las plantas sea escalable y sostenible a largo plazo. Este proyecto puede ser una solución asequible y eficaz para los productores y consumidores, lo que mejoraría la productividad y calidad de los cultivos y reduciría el impacto negativo en el medio ambiente.